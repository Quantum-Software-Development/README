{\rtf1\ansi\ansicpg1252\cocoartf2761
\cocoatextscaling0\cocoaplatform0{\fonttbl\f0\fnil\fcharset0 HelveticaNeue;\f1\fmodern\fcharset0 Courier;}
{\colortbl;\red255\green255\blue255;\red14\green14\blue14;\red240\green240\blue240;}
{\*\expandedcolortbl;;\cssrgb\c6667\c6667\c6667;\cssrgb\c95294\c95294\c95294;}
\paperw11900\paperh16840\margl1440\margr1440\vieww11520\viewh8400\viewkind0
\deftab720
\pard\pardeftab720\partightenfactor0

\f0\fs28 \cf2 \cb3 \expnd0\expndtw0\kerning0
\outl0\strokewidth0 \strokec2 '''\
\
In this formula, ( R ) represents the radius from the center of the torus tube to the center of the torus, ( r ) is the radius of the tube, and ( u, v ) are the parameters that vary along the angles of the torus. If you need any further explanation or help with LaTeX, I\'92m at your disposal.\
\
The hexadecimal code for GitHub blue is #0366d6. To apply this color in a LaTeX formula, you can use the\'a0
\f1 xcolor
\f0 \'a0package in LaTeX. Here is an example of how you can define the color and use it in the torus formula:\
""\
\
\
\\documentclass\{article\}\
\\usepackage\{xcolor\} % Package for colors\
\
\\definecolor\{githubblue\}\{HTML\}\{0366d6\} % Defining the GitHub blue color\
\
\\begin\{document\}\
\
\\[\
\\color\{githubblue\}\
\\begin\{align*\}\
x(u, v) &= (R + r \\cos v) \\cos u \\\\\
y(u, v) &= (R + r \\cos v) \\sin u \\\\\
z(u, v) &= r \\sin v\
\\end\{align*\}\
\\]\
\
\\end\{document\}\
}